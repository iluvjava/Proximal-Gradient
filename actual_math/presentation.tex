\documentclass[11pt]{beamer}
\usetheme{Madrid}
\usepackage[utf8]{inputenc}

\usepackage{hyperref}
\usepackage{amsmath}
\usepackage{amsfonts}
\usepackage{amssymb}
\usepackage{graphicx}
\DeclareMathOperator {\argmin}{argmin}
\usepackage{algorithmic}
\usepackage{algorithm}

\author{Hongda Li}
\title{Proximal Gradient: Convergence, Implementations and Applications}
% Informe o seu email de contato no comando a seguir
% Por exemplo, alcebiades.col@ufes.br
\newcommand{\email}{email}
\setbeamercovered{transparent}
\setbeamertemplate{navigation symbols}{}
%\logo{}
\institute[]{UBC Okanagan}
\date{\today}
\subject{MATH 590A}

% ---------------------------------------------------------
% Selecione um estilo de referência
\bibliographystyle{plain}

%\bibliographystyle{abbrv}
%\setbeamertemplate{bibliography item}{\insertbiblabel}
% ---------------------------------------------------------

% ---------------------------------------------------------
\newtheorem{remark}{Remark}
\newtheorem{assumption}{Assumption}

\begin{document}

\begin{frame}
    \titlepage
\end{frame}

\begin{frame}{ToC}
    \tableofcontents
\end{frame}

\section{Introduction and Prximal Operators}
    \subsection{Taxonomy of Proximal type of Methods}
        \begin{frame}{Sum of 2 Functions}
            \begin{align}
                \min_{x} g(x) + h(x)
            \end{align}
            Through out the presentation we assume that the objective of some kind of function $f$ can be interpreted as the sum of 2 functions.
            The paper we will be focusing on: FISTA (Fast Iterative-Shrinkage Algorithm) by Beck and Teboulle. 
            \begin{itemize}
                \item [1.] When $h = \delta_Q$ with $Q$ closed and convex with $Q\subseteq \text{ri}\circ \text{dom}(h)$, we use projected subgradient. 
                \item [2.] When $g$ is \textbf{\emph{strongly smooth}} and $h$ is \textbf{closed convex proper} whose proximal oracle is easy to compute, we consider the use of FISTA. 
                \item [3.] BIG Numerical Experiments!
            \end{itemize}
        \end{frame}
        \begin{frame}{Stuff to Go Over}
            \begin{block}{What is FISTA}
                Simply speaking, the FISTA algorithm is the non-smooth analogy of gradient descend with Nesterov Momentum.     
            \end{block}
            We will be going over these things in the presentations. 
            \begin{itemize}
                \item [1.] Derive the proximal gradient operator under standard convexity and regularity assumptions for the function $g, h$. 
                \item [2.] State one important lemma that arised during the proof for the proximal gradient method that is later useful for the proof for the FISTA. 
                \item [3.] Derive the FISTA algorithm's convergence rate and construct the sequence of the Nesterov Momentum during the proof using a template algorithm. 
            \end{itemize}
        \end{frame}
    \subsection{The Proximal Operator}
        \begin{frame}{Proximal Operator Definition}
            \begin{definition}{The Proximal Operator}
                Let $f$ be convex closed and proper, then the proximal operator paramaterized by $\alpha > 0$ is a non-expansive mapping defined as: 
                \begin{align*}
                    \text{prox}_{f, \alpha}(x) := 
                    \arg\min_{y}\left\lbrace
                        f(y) + \frac{1}{2\alpha} \Vert y - x\Vert^2
                    \right\rbrace. 
                \end{align*}
            \end{definition}  
        \end{frame}
        \begin{frame}{Prox is the Resolvant of Subgradient}
            \begin{lemma}[Resolvant of the Subgradient]\label{lemma:prox_alternative_form}
                When the function $f$ is convex closed and proper, the $\text{prox}_{\alpha, f}$ can be viewed as the following operator $(I + \alpha \partial f)^{-1}$. 
            \end{lemma}
            \begin{proof}
                \begin{align*}
                    \mathbf 0 &\in \partial
                    \left[
                        \left.
                            f(y) + \frac{1}{2\alpha} \Vert y - x\Vert^2 
                        \right| y
                    \right](y^+)
                    \\
                    \mathbf 0 &\in \partial f(y^+) + \frac{1}{\alpha}(y^+ - x)
                    \\
                    \frac{x}{\alpha} &\in 
                    (\partial f + \alpha^{-1}I)(y^+)
                    \\
                    x &\in 
                    (\alpha \partial f + I)(y^+)
                    \\
                    y &\in (\alpha\partial f+ I)^{-1}(x).
                \end{align*}
            \end{proof}
        \end{frame}
        \begin{frame}{An Example of Prox}
            \begin{definition}[Soft Thresholding]
                For some $x \in \mathbb R$, the proximal operator of it's absolute value is given as:
                \begin{align*}
                   \text{prox}_{\lambda \Vert\cdot \Vert_1, t}(x) = \text{sign}(x)\max(|x| - t\lambda , 0). 
                \end{align*}
            \end{definition}
            One could interpret the $\text{sign}$ operator as projecting $x$ onto the interval $[-1, 1]$ and the $\max(|x| - t\lambda , 0)$ as the distance of the point $x$ to the interval $[-t\lambda, t\lambda]$. 
        \end{frame}
        
    \subsection{Strong Smoothness}
        \begin{frame}{Strong Smoothness}
            \begin{definition}[Strong Smoothness]\label{def:strong_smoothness}
                A differentiable function $g$ is called strongly smooth with a constant $\alpha$ then it satisfies: 
                \begin{align}
                    |g(y) - g(x) - 
                    \langle \nabla g(x), y - x
                    \rangle| \le \frac{\alpha}{2}\Vert x - y\Vert^2
                    \quad \forall x, y\in \mathbb E. 
                \end{align}    
            \end{definition}
            \begin{block}{Remark}
                The absolute value sign can be removed and replaced with $0\le$ on the left when the function $g$ is a convex function.
            \end{block}
        \end{frame}
        \begin{frame}{Equivalence of Strong Smoothness and Lipschitz Gradient}
            \begin{theorem}[Lipschitz Gradient Equivalence under Convexity]
                Suppose $g$ is differentiable on the entire of $\mathbb E$. It is closed convex proper. It is strongly smooth with parameter $\alpha$ if and only if the gradient $\nabla g$ is globally Lipschitz continuous with a parameter of $\alpha$ and $g$ is closed and convex. 
                \begin{align*}
                    \Vert \nabla g(x) -\nabla g(y)\Vert \le 
                    \alpha 
                    \Vert x - y \Vert\quad \forall x, y\in \mathbb E
                \end{align*}
            \end{theorem}
            
        \end{frame}
    \subsection{A Major Assumption}    
        \begin{frame}{A Major Assumption}
            \begin{assumption}[Convex Smooth Nonsmooth with Bounded Minimizers]\label{assumption:1}
                We will assume that $g:\mathbb E\mapsto \mathbb R$ is \textbf{strongly smooth} with constant $L_g$ and $h:\mathbb E \mapsto \bar{\mathbb R}$ \textbf{is closed convex and proper}. We define $f := g + h$ to be the summed function and $\text{ri}\circ \text{dom}(g) \cap \text{ri}\circ \text{dom}(h) \neq \emptyset$. We also assume that a set of minimizers exists for the function $f$ and that the set is bounded. Denote the minimizer using $\bar x$. 
            \end{assumption}
        \end{frame}
        
    
\section{Envelope and Prox 2 Points}
    \begin{frame}{Envelope and Upper Bounding Functions}
        \begin{block}{Upper Bounding Function}
            With assumption 1, we construct an upper bounding functions at the point $x$ evaluated at $y$ for the function $f$ and it's given by: 
            \begin{align*}
                g(x) + \nabla g(x)^T(y - x) + \frac{\beta}{2} \Vert y - x\Vert^2
                + h(y) =: m_x(y|\beta) \quad \forall y \in \mathbb E, 
            \end{align*}    
        \end{block}
        In brief, suppose we are at the point $x$ of the iterations we are minimizing the function $m_x(y|\beta)$ to obtain the next point for our iterations.
        \begin{theorem}[Minimizer of the Envelope]\label{thm:minimizer_envelope}
            The minimizer for the envelope has a closed form, and it is $\text{prox}_{h, \beta^{-1}}(x + \beta^{-1}\nabla g(x))$, with \hyperref[assumption:1]{assumption \ref*{assumption:1}}. 
        \end{theorem}
    \end{frame}
    \begin{frame}{The Prox Gradient Operator}
        \begin{proof}{Minimizer of the Envelope}
            We consider minimizing the envelope; zero is in the subgradient of the upper bounding function $m_x(y|\beta)$. 
            \begin{align*}
                \mathbf 0 &\in 
                \nabla g(x) + {\beta}(y - x) + \partial h(y)
                \\
                \nabla g(x) + \beta x & \in
                \beta y + \partial h(y)
                \\
                -\beta^{-1} \nabla g(x) + x &\in y + \beta^{-1} \partial h(y)
                \\
                -\beta^{-1} \nabla g(x) + x &\in [I + \beta^{-1} \partial h](y)
                \\
                \implies
                [I + \beta^{-1}\partial h]^{-1}(- \beta^{-1} \nabla g(x) + x) 
                & \ni y,
            \end{align*}
            recall \hyperref[lemma:prox_alternative_form]{lemma \ref*{lemma:prox_alternative_form}}, it's the operator $\text{prox}_{h, \beta^{-1}}(x + \beta^{-1}\nabla g(x))$. 
        \end{proof}
    \end{frame}
    \begin{frame}{Prox Step and the Proximal Gradient Algorithm}
        \begin{block}{The Prox Step}
            For simplicty we will be calling the point $\text{prox}_{h, \beta^{-1}}(x + \beta^{-1}\nabla g(x))$ ``the prox step'', and we denote it as $\mathcal P_{\beta^{-1}}^{g, h}(x)$ when there is no ambiguity we simply use $\mathcal Px$.     
        \end{block}
        \begin{block}{The Proximal Gradient Method}
            \begin{algorithm}[H]
                \algsetup{linenosize=\tiny}
                \scriptsize
                \begin{algorithmic}[1]
                \STATE{\textbf{Input:} $g, h$, smooth and nonsmooth, $L$ stepsize, $x^{(0)}$ an initial guess of solution. }
                \FOR{$k=1, 2,\cdots, N$}
                    \STATE{\quad $x^{(k + 1)} = \mathcal P_L^{g, h}x^{(k)}$}
                    \IF{$x^{(k + 1)}, x^{(k)}$ close enough}
                        \STATE{\textbf{Break}}
                    \ENDIF
                \ENDFOR
                \end{algorithmic}
                \caption{Proximal Gradient With Fixed Step-sizes}
                \label{alg:1}
            \end{algorithm}
        \end{block}
    \end{frame}
    \begin{frame}{The proximal Gradient Method}
        \begin{itemize}
            \item [1.] Converges Monotonically for stepsize $L \ge L_g$. 
            \item [2.] It has a convergence rate of $\mathcal O(1/k)$ on the optimality gap $\Delta_k := f(x^{(k)}) - f(\bar x)$ where $\bar x$ is one of the minimizers for $f$ satisfiying \hyperref[assumption:1]{assumption \ref*{assumption:1}}. 
            \item [3.] Line search routine can be applied and the stepsize should satisfies the conditions that: 
        \end{itemize}
    \end{frame}


   

\section{References}
    \begin{frame}{References}
        \bibliography{bib.tex}
    \end{frame}

\end{document}